
\documentclass[11pt]{article}

% Paquetes
%===================================================================================================

% Establecemos los márgenes
\usepackage[a4paper, margin=1in]{geometry}

% Separacion entre parrafos
\setlength{\parskip}{1em}

% Paquete para incluir codigo
\usepackage{listings}

% Paquete para incluir imagenes
\usepackage{graphicx}
\graphicspath{ {./images/} }

% Para fijar las imagenes en la posicion deseada
\usepackage{float}

% Para que el codigo acepte caracteres en utf8
\lstset{literate=
  {á}{{\'a}}1 {é}{{\'e}}1 {í}{{\'i}}1 {ó}{{\'o}}1 {ú}{{\'u}}1
  {Á}{{\'A}}1 {É}{{\'E}}1 {Í}{{\'I}}1 {Ó}{{\'O}}1 {Ú}{{\'U}}1
  {à}{{\`a}}1 {è}{{\`e}}1 {ì}{{\`i}}1 {ò}{{\`o}}1 {ù}{{\`u}}1
  {À}{{\`A}}1 {È}{{\'E}}1 {Ì}{{\`I}}1 {Ò}{{\`O}}1 {Ù}{{\`U}}1
  {ä}{{\"a}}1 {ë}{{\"e}}1 {ï}{{\"i}}1 {ö}{{\"o}}1 {ü}{{\"u}}1
  {Ä}{{\"A}}1 {Ë}{{\"E}}1 {Ï}{{\"I}}1 {Ö}{{\"O}}1 {Ü}{{\"U}}1
  {â}{{\^a}}1 {ê}{{\^e}}1 {î}{{\^i}}1 {ô}{{\^o}}1 {û}{{\^u}}1
  {Â}{{\^A}}1 {Ê}{{\^E}}1 {Î}{{\^I}}1 {Ô}{{\^O}}1 {Û}{{\^U}}1
  {ã}{{\~a}}1 {ẽ}{{\~e}}1 {ĩ}{{\~i}}1 {õ}{{\~o}}1 {ũ}{{\~u}}1
  {Ã}{{\~A}}1 {Ẽ}{{\~E}}1 {Ĩ}{{\~I}}1 {Õ}{{\~O}}1 {Ũ}{{\~U}}1
  {œ}{{\oe}}1 {Œ}{{\OE}}1 {æ}{{\ae}}1 {Æ}{{\AE}}1 {ß}{{\ss}}1
  {ű}{{\H{u}}}1 {Ű}{{\H{U}}}1 {ő}{{\H{o}}}1 {Ő}{{\H{O}}}1
  {ç}{{\c c}}1 {Ç}{{\c C}}1 {ø}{{\o}}1 {å}{{\r a}}1 {Å}{{\r A}}1
  {€}{{\euro}}1 {£}{{\pounds}}1 {«}{{\guillemotleft}}1
  {»}{{\guillemotright}}1 {ñ}{{\~n}}1 {Ñ}{{\~N}}1 {¿}{{?`}}1 {¡}{{!`}}1
}

% Para que no se salgan las lineas de codigo
% Para fijar una fuente que resalte
\lstset{breaklines=true, basicstyle=\ttfamily}

% Para que los metadatos que escribe latex esten en español
\usepackage[spanish]{babel}
\decimalpoint % Para que no se cambie el punto a la coma

% Para la bibliografia
% Sin esto, los enlaces de la bibliografia dan un error de compilacion
\usepackage{url}

% Para que se puedan clicar los enlaces
\usepackage{hyperref}

% Para mostrar graficas de dos imagenes, cada una con su caption, y con un caption comun
\usepackage{subcaption}

% Simbolo de los numeros reales
\usepackage{amssymb}

% Para que los codigos tengan una fuente distinta
\usepackage{courier}

\lstdefinestyle{CustomStyle}{
  language=Python,
  numbers=left,
  stepnumber=1,
  numbersep=10pt,
  tabsize=4,
  showspaces=false,
  showstringspaces=false
  basicstyle=\tiny\ttfamily,
}

% Para referenciar secciones usando el nombre de las secciones
\usepackage{nameref}

% Para enumerados dentro de enumerados
\usepackage{enumitem}

% Para mejores tablas
\usepackage{tabularx}

% Para poder tener el mismo identificador en dos tablas separadas
\usepackage{caption}

% Mostrar la página de las referencias en el indice del documento
\usepackage[nottoc,numbib]{tocbibind}

% Para mostrar las matrices
\usepackage{amsmath}

% Para que las notas al pie de pagina queden bien abajo
\usepackage[bottom]{footmisc}

% Para poner tablas en horizontal, ocupando bien la página
% cuando hay mucho texto en la table
\usepackage{lscape}

% Comandos personalizados
%===================================================================================================
% Here all custom commands are defined

% Para realizar las citas de forma corta
\newcommand{\customcite}[1]{\emph{"\ref{#1}. \nameref{#1}"}}

% Para entrecomillar un texto
\newcommand{\entrecomillado}[1]{\emph{``#1''}}




% Metadatos del documento
%===================================================================================================
\title{
    Estudio sobre la empleabilidad de los estudiantes
}

\author{
    {Sergio Quijano Rey}\\
    {sergioquijano@correo.ugr.es}
}

\date{\today}

% Separacion entre parrafos
\setlength{\parskip}{1em}

% Contenido del documento
%===================================================================================================
\begin{document}

% Portada del documento
\maketitle
\pagebreak

% Indice de contenidos
\tableofcontents

% Lista de figuras
% Uso el addtocontents para que no se muestre la seccion de indice de figuras en el indice inicial

\addtocontents{toc}{\setcounter{tocdepth}{-10}}
\listoffigures

\listoftables

% \lstlistoflistings
\addtocontents{toc}{\setcounter{tocdepth}{3}}

\pagebreak

% Contents of the document

\section{Abstract}

En este trabajo, estudiaremos una base de datos consistente en métricas recogidas durante entrevistas de prueba, junto a si los candidatos son o no escogidos para el hipotético puesto de trabajo.

Con \textbf{dos objetivos} en mente:

\begin{itemize}
    \item Construir un clasificador eficiente para predecir la empleabilidad
    \item Usar la base de datos para realizar un estudio sobre lo meritocrático del proceso. Dicho experimento consiste en entrenar dos modelos, uno usando las variables que consideramos meritocráticas, y otro usando las variables que consideramos no meritocráticas. Si el modelo que usa variables no meritocráticas funciona mejor, podemos pensar que entonces el proceso de selección se basa más en estos aspectos, que hemos considerado no meritocráticos
\end{itemize}

Para ello, realizaremos:

\begin{itemize}
    \item Un estudio univariante de la base de datos, destacando el tratamiento de \textit{outliers}, el estudio de la normalidad univariante, y un análisis descriptivo clásico
    \item Un estudio multivariante de la base de datos, destacando el estudio de las correlaciones entre variables, tratamiento multivariante de \textit{outliers}, estudio de normalidad multivariante y reducción de la dimensionalidad con \textit{PCA} y \textit{FA}
    \item Ajuste de los modelos, destacando una pequeña exploración de hiperparámetros, entrenamiento y validación, comparando los resultados
    \item El experimento adicional, previamente mencionado
\end{itemize}

Al final, conseguimos obtener un modelo muy robusto a la hora de realizar predicciones, y todo el análisis realizado en el cuaderno, más el experimento adicional, confirman de forma contundente la falta de meritocracia en el proceso de selección.




\newpage

\section{introduccion}

TODO


% TODO -- Último párrafo diciendo cuáles han sido los objetivos del trabajo
% TODO -- Le da relevancia a que comentemos el estado del arte
% TODO -- Importancia a referencias a otros trabajos

\newpage

\section{materialesmetodos}

TODO


% TODO -- Describir la base de datos que usamos
% TODO -- Los métodos no deben ser más de 400 palabras, describiendo los métodos estadísticos que hemos usado

TODO -- comentar que la base de datos usada se encuentra en \cite{database:online}

\newpage

\section{Resultados}



\newpage

\section{discusion} \label{section:Discusion}

TODO


% TODO -- Comentar, interpretar y explicar lo que significan los resultados


En el cuaderno que se entrega junto a esta memoria, se discuten prácticamente todos los resultado que obtenemos más o menos en profundidad. Por tanto, aquí solo comentamos los resultados más relevantes. Si el lector tiene curiosidad por profundizar en alguno de los resultados presentados previamente, en dicho cuaderno seguramente encuentre dichos resultados discutidos.

El \textbf{primero de nuestros objetivos era construir un clasificador robusto}. Podemos considerar que hemos logrado este objetivo. En \ref{section:hiperparametros} elegimos \textit{XGBOOST} sobre el conjunto de datos al que aplicamos \textit{PCA} como el modelo más robusto. En \ref{section:validacion} vemos que:

\begin{itemize}
    \item Efectivamente, \textit{XGBOOST} es el modelo que mejor funciona para esta base de datos
    \item El modelo generaliza bien, puesto que en el conjunto de \textit{test} se comporta prácticamente igual de bien
\end{itemize}

Era de esperar que \textit{XGBOOST} fuera el mejor modelo. Los supuestos sobre los que se fundamentan los dos modelos de discrimante fallan, y \textit{XGBOOST} es de los modelos más potentes para datos tabulares.

Nuestro \textbf{segundo objetivo era comprobar si el proceso de selección no es meritocrático}. Esta pregunta es más complicada de responder, y nos tenemos que fundamentar en muchos más recursos.

En primer lugar, vemos en \ref{figure:matriz_correlaciones} que todas las variables de entrada, salvo el \textbf{rendimiento académico}, están bastante correlacionadas entre sí. El rendimiento académico apenas está correlacionada con ninguna otra variable. Esto nos hace sospechar en un primer momento que \textbf{dicha variable no va a ser relevante} a la hora de determinar la empleabilidad. De hecho, es la variable menos correlacionada con la variable de salida. Es más, en \ref{figure:pares_variables_mas_correlacionadas} no aparece. En dicho gráfico, y en \ref{figure:pares_variables_mas_correlacionadas_segun_empleabilidad}, podemos ver que la apariencia y la confianza son de los conceptos más relevantes, considerando que ambos son (según nuestro criterio) no meritocráticos.

Los dos métodos de reducción de dimensionalidad dejan claro que el rendimiento es muy poco relevante. En ambos casos, el primer elemento (componente principal o variable latente) se compone de una combinación de todas las variables menos el rendimiento. El segundo elemento deja al rendimiento en solitario, teniendo mucha menos importancia dicho segundo elemento. Esto se fundamenta en \ref{figure:pca_ejes} y \ref{figure:fa_combinacion_variables}

Con \ref{figure:coeficientes_lda} obtenemos un modelo más justo de lo que esperábamos: la agilidad mental, formas de hablar y habilidad para presentar ideas son las tres variables más importantes, con una gran diferencia respecto a la cuarta (la confianza). Tenemos dos variables meritocráticas (agilidad mental y habilidad para presentar ideas) como las más relevantes. Aunque hay que comentar que este modelo no obtiene resultados satisfactorios. En \ref{figure:importancias_xgb} las dos variables más importantes son la agilidad mental y la apariencia. Nos sorprende que \textbf{la segunda variable más importante a la hora de determinar la empleabilidad}, según este modelo, sea la \textbf{apariencia}, que quizás pueda considerarse la \textbf{variable menos meritocrática de todas}, solo por debajo de la condición física. En tercer lugar tenemos el rendimiento académico. Nos sorprende porque ya hemos visto la poca relevancia que parece tener. Sin embargo, seguimos pensando que tiene poca relevancia. Al trabajar con árboles de decisión, puede ocurrir que usemos esta variable como desempate cuando la agilidad mental y apariencia no sea suficientes para determinar la empleabilidad.

El experimento deja claro que el proceso no es meritocrático, por lo que vemos en \ref{table:resultados_experimento}. Aunque la diferencia en \textit{test} sea pequeña, estamos superando el rendimiento usando \textbf{únicamente variables no meritocráticas}. Con solo conocer las formas de hablar, condición física, apariencia y confianza obtenemos mejores resultados que conociendo el rendimiento académico, las habilidades de comunicación y de presentar ideas y la agilidad mental, que deberían ser mucho más útiles.

\newpage

\section{conclusion}

TODO

\newpage

\pagebreak


% Which style to use to show references
\bibliographystyle{ieeetr}

% Show the references
\bibliography{./References}

\end{document}
