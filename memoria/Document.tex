
\documentclass[11pt]{article}

% Paquetes
%===================================================================================================

% Establecemos los márgenes
\usepackage[a4paper, margin=1in]{geometry}

% Separacion entre parrafos
\setlength{\parskip}{1em}

% Paquete para incluir codigo
\usepackage{listings}

% Paquete para incluir imagenes
\usepackage{graphicx}
\graphicspath{ {./images/} }

% Para fijar las imagenes en la posicion deseada
\usepackage{float}

% Para que el codigo acepte caracteres en utf8
\lstset{literate=
  {á}{{\'a}}1 {é}{{\'e}}1 {í}{{\'i}}1 {ó}{{\'o}}1 {ú}{{\'u}}1
  {Á}{{\'A}}1 {É}{{\'E}}1 {Í}{{\'I}}1 {Ó}{{\'O}}1 {Ú}{{\'U}}1
  {à}{{\`a}}1 {è}{{\`e}}1 {ì}{{\`i}}1 {ò}{{\`o}}1 {ù}{{\`u}}1
  {À}{{\`A}}1 {È}{{\'E}}1 {Ì}{{\`I}}1 {Ò}{{\`O}}1 {Ù}{{\`U}}1
  {ä}{{\"a}}1 {ë}{{\"e}}1 {ï}{{\"i}}1 {ö}{{\"o}}1 {ü}{{\"u}}1
  {Ä}{{\"A}}1 {Ë}{{\"E}}1 {Ï}{{\"I}}1 {Ö}{{\"O}}1 {Ü}{{\"U}}1
  {â}{{\^a}}1 {ê}{{\^e}}1 {î}{{\^i}}1 {ô}{{\^o}}1 {û}{{\^u}}1
  {Â}{{\^A}}1 {Ê}{{\^E}}1 {Î}{{\^I}}1 {Ô}{{\^O}}1 {Û}{{\^U}}1
  {ã}{{\~a}}1 {ẽ}{{\~e}}1 {ĩ}{{\~i}}1 {õ}{{\~o}}1 {ũ}{{\~u}}1
  {Ã}{{\~A}}1 {Ẽ}{{\~E}}1 {Ĩ}{{\~I}}1 {Õ}{{\~O}}1 {Ũ}{{\~U}}1
  {œ}{{\oe}}1 {Œ}{{\OE}}1 {æ}{{\ae}}1 {Æ}{{\AE}}1 {ß}{{\ss}}1
  {ű}{{\H{u}}}1 {Ű}{{\H{U}}}1 {ő}{{\H{o}}}1 {Ő}{{\H{O}}}1
  {ç}{{\c c}}1 {Ç}{{\c C}}1 {ø}{{\o}}1 {å}{{\r a}}1 {Å}{{\r A}}1
  {€}{{\euro}}1 {£}{{\pounds}}1 {«}{{\guillemotleft}}1
  {»}{{\guillemotright}}1 {ñ}{{\~n}}1 {Ñ}{{\~N}}1 {¿}{{?`}}1 {¡}{{!`}}1
}

% Para que no se salgan las lineas de codigo
% Para fijar una fuente que resalte
\lstset{breaklines=true, basicstyle=\ttfamily}

% Para que los metadatos que escribe latex esten en español
\usepackage[spanish]{babel}
\decimalpoint % Para que no se cambie el punto a la coma

% Para la bibliografia
% Sin esto, los enlaces de la bibliografia dan un error de compilacion
\usepackage{url}

% Para que se puedan clicar los enlaces
\usepackage{hyperref}

% Para mostrar graficas de dos imagenes, cada una con su caption, y con un caption comun
\usepackage{subcaption}

% Simbolo de los numeros reales
\usepackage{amssymb}

% Para que los codigos tengan una fuente distinta
\usepackage{courier}

\lstdefinestyle{CustomStyle}{
  language=Python,
  numbers=left,
  stepnumber=1,
  numbersep=10pt,
  tabsize=4,
  showspaces=false,
  showstringspaces=false
  basicstyle=\tiny\ttfamily,
}

% Para referenciar secciones usando el nombre de las secciones
\usepackage{nameref}

% Para enumerados dentro de enumerados
\usepackage{enumitem}

% Para mejores tablas
\usepackage{tabularx}

% Para poder tener el mismo identificador en dos tablas separadas
\usepackage{caption}

% Mostrar la página de las referencias en el indice del documento
\usepackage[nottoc,numbib]{tocbibind}

% Para mostrar las matrices
\usepackage{amsmath}

% Para que las notas al pie de pagina queden bien abajo
\usepackage[bottom]{footmisc}

% Para poner tablas en horizontal, ocupando bien la página
% cuando hay mucho texto en la table
\usepackage{lscape}

% Comandos personalizados
%===================================================================================================
% Here all custom commands are defined

% Para realizar las citas de forma corta
\newcommand{\customcite}[1]{\emph{"\ref{#1}. \nameref{#1}"}}

% Para entrecomillar un texto
\newcommand{\entrecomillado}[1]{\emph{``#1''}}




% Metadatos del documento
%===================================================================================================
\title{
    Estudio sobre la empleabilidad de los estudiantes
}

\author{
    {Sergio Quijano Rey}\\
    {sergioquijano@correo.ugr.es}
}

\date{\today}

% Separacion entre parrafos
\setlength{\parskip}{1em}

% Contenido del documento
%===================================================================================================
\begin{document}

% Portada del documento
\maketitle
\pagebreak

% Indice de contenidos
\tableofcontents

% Lista de figuras
% Uso el addtocontents para que no se muestre la seccion de indice de figuras en el indice inicial

\addtocontents{toc}{\setcounter{tocdepth}{-10}}
\listoffigures

\listoftables

% \lstlistoflistings
\addtocontents{toc}{\setcounter{tocdepth}{3}}

\pagebreak

% Contents of the document

\section{Abstract}

En este trabajo, estudiaremos una base de datos consistente en métricas recogidas durante entrevistas de prueba, junto a si los candidatos son o no escogidos para el hipotético puesto de trabajo.

Con \textbf{dos objetivos} en mente:

\begin{itemize}
    \item Construir un clasificador eficiente para predecir la empleabilidad
    \item Usar la base de datos para realizar un estudio sobre lo meritocrático del proceso. Dicho experimento consiste en entrenar dos modelos, uno usando las variables que consideramos meritocráticas, y otro usando las variables que consideramos no meritocráticas. Si el modelo que usa variables no meritocráticas funciona mejor, podemos pensar que entonces el proceso de selección se basa más en estos aspectos, que hemos considerado no meritocráticos
\end{itemize}

Para ello, realizaremos:

\begin{itemize}
    \item Un estudio univariante de la base de datos, destacando el tratamiento de \textit{outliers}, el estudio de la normalidad univariante, y un análisis descriptivo clásico
    \item Un estudio multivariante de la base de datos, destacando el estudio de las correlaciones entre variables, tratamiento multivariante de \textit{outliers}, estudio de normalidad multivariante y reducción de la dimensionalidad con \textit{PCA} y \textit{FA}
    \item Ajuste de los modelos, destacando una pequeña exploración de hiperparámetros, entrenamiento y validación, comparando los resultados
    \item El experimento adicional, previamente mencionado
\end{itemize}

Al final, conseguimos obtener un modelo muy robusto a la hora de realizar predicciones, y todo el análisis realizado en el cuaderno, más el experimento adicional, confirman de forma contundente la falta de meritocracia en el proceso de selección.




\newpage

\section{introduccion}

TODO


% TODO -- Último párrafo diciendo cuáles han sido los objetivos del trabajo
% TODO -- Le da relevancia a que comentemos el estado del arte
% TODO -- Importancia a referencias a otros trabajos

\newpage

\section{materialesmetodos}

TODO


% TODO -- Describir la base de datos que usamos
% TODO -- Los métodos no deben ser más de 400 palabras, describiendo los métodos estadísticos que hemos usado

TODO -- comentar que la base de datos usada se encuentra en \cite{database:online}

\newpage

\section{Resultados}



\newpage

\section{discusion} \label{section:Discusion}

TODO


% TODO -- Comentar, interpretar y explicar lo que significan los resultados

\newpage

\section{conclusion}

TODO

\newpage

\pagebreak


% Which style to use to show references
\bibliographystyle{ieeetr}

% Show the references
\bibliography{./References}

\end{document}
